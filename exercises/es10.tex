\begin{exercise}{
    Prove or disprove the following semantic equivalence:
    \[ \wrepeat{(\wrepeat{S}{b})}{b} \DScong \wrepeat{S}{b} \]
}
    We want to prove:
    \[ \Sds{\wrepeat{(\wrepeat{S}{b})}{b}} = \Sds{\wrepeat{S}{b}} \]
    That is, $\FIX F = \FIX G$ where:
    \begin{align*}
        F\ g &= \cond{\Bool{b}}{\id}{g} \circ \FIX G \\
        G\ g &= \cond{\Bool{b}}{\id}{g} \circ \Sds{S}
    \end{align*}
    We first prove $(\FIX G)\ s = s' \neq \undef \implies \Bool{b} s' = \True$, that is the termination lemma for $\wrepeat{S}{b}$.
    Consider the following function:
    \[
        d\ s = \begin{cases}
            \undef &\quad \text{if } (\FIX G)\ s = s' \neq \undef, \Bool{b} s' = \False \\
            (\FIX G)\ s &\quad \text{otherwise}
        \end{cases}
    \]
    We prove that $G d \sqsubseteq d$, that is:
    \[ \forall s, s' \in \State.\ (G\ d)\ s = s' \implies d\ s = s' \]
    \[ (G\ d)\ s = (\cond{\Bool{b}}{\id}{d} \circ \Sds{S})\ s = s' \]
    By the definition of function composition $\exists s'' \in \State.\ \Sds{S}\ s = s'' \neq \undef,$ \\ $\cond{\Bool{b}}{\id}{d} s'' = s'$
    We distinguish two cases:
    \begin{itemize}
        \item $\Bool{b} s'' = \True$
            \[ (G\ d)\ s = \cond{\Bool{b}}{\id}{d} s'' = \id s'' = s'' = s' \]
            Thus $\Bool{b} s' = \Bool{b} s'' = \True$
            \[ (\FIX G)\ s = G(\FIX G)\ s = (\cond{\Bool{b}}{\id}{\FIX G} \circ \Sds{S})\ s = \id s'' = s'' = s' \]
            Since $(\FIX G)\ s = s' \neq \undef$ and $\Bool{b} s' \neq \False$ we have $d\ s = (\FIX G)\ s = s'$
        \item $\Bool{b} s'' = \False$
            \[ (G\ d)\ s = \cond{\Bool{b}}{\id}{d} s'' = d\ s'' \]
            \begin{align*}
                (\FIX G)\ s &= G(\FIX G)\ s = (\cond{\Bool{b}}{\id}{\FIX G} \circ \Sds{S})\ s = \\
                &= \cond{\Bool{b}}{\id}{\FIX G}\ s'' = \FIX G\ s''
            \end{align*}
            We distinguish two cases:
            \begin{itemize}
                \item $(\FIX G)\ s = (\FIX G)\ s'' = s''' \neq \undef, \Bool{b} s''' = \False$, then $(G\ d)\ s = d\ s'' = \undef = d\ s$
                \item otherwise, $(G\ d)\ s = d\ s'' = (\FIX G)\ s'' = (\FIX G)\ s = d\ s$
            \end{itemize}
    \end{itemize}
    We already proved in class that such functional $G$ is continuous.
    Thus by the fixpoint induction lemma we have $\FIX G \sqsubseteq d$. which means that:
    \[ \forall s, s' \in \State.\ (\FIX G)\ s = s' \implies d\ s = s' \]
    Assume by contradiction that $\exists s, s' \in \State.\ (\FIX G)\ s = s', \Bool{b} s' = \False $. Then $d\ s = \undef \neq s'$, which means $\FIX G \not\sqsubseteq d$, and that's a contradiction. Thus $(\FIX G)\ s = s' \implies \Bool{b} s' = \True$ \\
    Now we can finally prove $\FIX F = \FIX G$, that is:
    \[ \forall s, s' \in \State.\ (\FIX F)\ s = s' \iff (\FIX G)\ s = s' \]
    \begin{itemize}
        \item $(\implies)$
            \[ (\FIX F)\ s = F(\FIX F)\ s = (\cond{\Bool{b}}{\id}{\FIX F} \circ \FIX G)\ s \]
            By the definition of composition $\exists s'' \in \State.\ (\FIX G)\ s = s'' \neq \undef$ \\
            By the termination lemma we just proved we get $\Bool{b} s'' = \True$, then:
            \[ \cond{\Bool{b}}{\id}{\FIX F}\ s'' = \id s'' = s'' = s' \]
            And so $(\FIX G)\ s = s'' = s'$
        \item $(\impliedby)$ \\
            By the termination lemma we just proved we get $\Bool{b} s = \True$, then:
            \begin{align*}
                (\FIX F)\ s &= F(\FIX F)\ s = (\cond{\Bool{b}}{\id}{\FIX F} \circ \FIX G)\ s = \\
                &= \cond{\Bool{b}}{\id}{\FIX F}\ s' = \id s' = \\
                &= s'
            \end{align*}
    \end{itemize}
\end{exercise}
